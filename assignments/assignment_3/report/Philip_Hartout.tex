\documentclass[a4paper,10pt,twoside]{article}
\usepackage{bera}
\renewcommand{\familydefault}{\sfdefault}
\usepackage[T1]{fontenc}
\usepackage[utf8x]{inputenc}
\usepackage[english]{babel} 
\usepackage[usenames,dvipsnames,figures]{xcolor}
\usepackage{lmodern,tikz,makeidx,graphicx,eurosym,amsmath,comment,titlesec,subcaption,textgreek,multirow,hyperref,url,enumerate,chemfig,sectsty,mathrsfs,amssymb,amsthm,multicol,fancyhdr,setspace,indentfirst,multicol,gensymb,textcomp, booktabs}
\usepackage[neverdecrease]{paralist}
\usepackage[final]{pdfpages}
\usepackage[top=3cm, bottom=3cm, left=3cm, right=3cm]{geometry}

\hypersetup{
    colorlinks=false,       % false: boxed links; true: colored links
    linkcolor=black,          % color of internal links (change box color with linkbordercolor)
    citecolor=black,        % color of links to bibliography
    filecolor=black,      % color of file links
    urlcolor=black           % color of external links
}

\setlength\columnsep{26pt}

\pagestyle{fancy}
\setlength{\headheight}{25pt} 
\lhead{Philip Hartout}
\rhead{Computational Biology Assignment 3}
\rfoot{}

\setlength\columnsep{26pt}

\hypersetup{
        linkcolor=blue,  
        colorlinks=true,
        citecolor=blue,   
        urlcolor=black,
}

\newcommand{\HRule}{\rule{\linewidth}{0.5mm}} 
\newlength{\drop}


\title{Computational Biology \\ Assignment 3 - Report}
\author{Philip Hartout \\ \url{phartout@student.ethz.ch}}
\date{\today}

\begin{document}

\maketitle

\begin{enumerate}
    \item If we take two very different sequences, we get an undefined distance metric, because in the indicated case, $V=\frac{1}{3}$ and $S=\frac{2}{3}$. In turn, $\frac{1}{2}log(1-2\cdot S-V)-\frac{1}{4}\cdot log(1-2\cdot V)$ becomes undefined because the log of negative values does not exist.
    \item My implementation of the UPGMA algorithm will probably be influenced by the order in which the sequences are given, because my implementation can only merge one node at once, and in the case of multiple minimal values in the distance matrix, this will yield the first such smallest value. 
    \item There are several features that may contribute to such a discrepancy:
    \begin{enumerate}
        \item It is likely that true evolution does not follow a strict molecular clock
        \item It can also be that the sequences have not been sampled at the same evolutionary time.
    \end{enumerate}
    \item Non-ultrametric trees construcuted by means of the neighbour-joining algorithm allow for rate changes in substitutions, which can remedy the fact that the strict molecular clock assumption is violated.  
\end{enumerate}

\end{document}